
\documentclass{resume} 

\usepackage[left=0.25in,top=0.2in,right=0.25in,bottom=0.2in]{geometry} 
\newcommand{\tab}[1]{\hspace{.2667\textwidth}\rlap{#1}}
\newcommand{\itab}[1]{\hspace{0em}\rlap{#1}}
\name{Mingyang (David) Pan} 
\address{1200 S 54th Street, Richmond, CA 94804 \\ davidpan.co.uk}
\address{909-859-5850 \\ \textbf{ympan99@gmail.com} \\ \textbf{github.com/dwpannn }\\ linkedin.com/in/wwdqd} 



\begin{document}

%----------------------------------------------------------------------------------------
%	EDUCATION SECTION
%----------------------------------------------------------------------------------------

\begin{rSection}{Education}

{\bf University of California, Berkeley} \hfill {\em May 2018 - \textbf{May 2021}} 
\\ Computer Science, B.A.


\end{rSection}
%----------------------------------------------------------------------------------------
%	TECHNICAL SKILLS SECTION
%----------------------------------------------------------------------------------------

\begin{rSection}{Technical Skills}

\begin{tabular}{ @{} >{\bfseries}l @{\hspace{6ex}} l }
Programming Languages &  \textbf{Java}, J2EE(Java Platform Enterprise Edition), Ruby(on Rails), \textbf{Python}, C \\
Amazon Web Services(AWS) & AWS Certified Developer-Associate \emph{(Verification\#: ZD27HPCKEFQ11R33)} \\
Back-end Development & REST \textbf{API consume, MySQL, NoSQL}, Dynamo DB, AWS RDS 
\end{tabular}

\end{rSection}

%----------------------------------------------------------------------------------------
%	WORK EXPERIENCE SECTION
%----------------------------------------------------------------------------------------

\begin{rSection}{Work Experience}

\begin{rSubsection}{XRCloud.net Inc}{\em May 2019 - August 2019}{\textbf{Software Engineering Intern - Cloud Solution}}{Los Angeles, California}{}
\item Used both the DevOp(software development and information-technology operations) Model and the CI/CD(Continuous Integration/Continuous Delivery) production pipeline to develop and maintain web applications.
\item Configured SQL servers with VMware vCenter and constructed Cloud Network with Apache Cloudstack.
\item Decreased edge location read latency for web applications by 37.9\% (based on AWS CloudWatch metrics) using read replicas and lambda functions.
\end{rSubsection}


%------------------------------------------------

\begin{rSubsection}{UC Berkeley EECS Department}{\em January 2019 - May 2019}{\textbf{Academic Intern}}{Berkeley, California}{}
\item Provided logistical support for CS 61B (Data Structure) course with over 1500 enrolled students.
\item Academically support students in lab sections and office hours three hours a week.
\item Logged 35.5 hours of work over the Spring 2019 semester. 
\end{rSubsection}

\end{rSection}

\begin{rSection}{Selected Projects}

\begin{rSubsection}{My personal website: davidpan.co.uk
}{\em August 2019}{\textbf{GitHub: github.com/dwpannn/myWeb}}{}
\item A simple portfolio-style website I built using JavaScript CSS, HTML, and bootstrap template. The content has been made highly available across the globe by leveraging AWS Cloudfront resources and using auto-scaling group. Incoming traffic volume has increased 17.4\% in the first month and average latency has dropped by 43.3\%. 
\end{rSubsection}

%------------------------------------------------

\begin{rSubsection}{Web application demonstrating one to two bit oblivious transfer theorem.}{\em February 2019 - May 2019}{\textbf{GitHub: github.com/dwpannn/oblivious-theorem}}{}
\item A website based on Ruby on Rails to solve the issue that sometimes people feel awkward when confessing or talking to others' about personal feelings. Namely, this prototype solves the problem of not knowing whether your crush likes you or not. The logic behind the operation is one to two bit oblivious transfer theorem. 
\end{rSubsection}

%------------------------------------------------

\begin{rSubsection}{Java Application based on refined Huffman coding algorithm}{\em March 2019}{\textbf{GitHub: github.com/dwpannn/huffmanCoding}}{}
\item An application uses truncated tree structure to shorten the length of the binary string returned. Increased encoding efficiency over traditional tree structure by over 25\%.
\end{rSubsection}

\end{rSection}
\begin{rSection}{Relevant Courses}
\\ \itab{The Structure and Interpretation of Computer Programs} \tab{}  \tab{\textbf{Introduction to Database Systems}}
\\ \itab{\textbf{Data Structures}} \tab{}  \tab{Directed Group Studies for Advanced Undergraduates} 
\\ \itab{Great Ideas of Computer Architecture (Machine Structures)} \tab{}  \tab{Adaptive Instruction Methods in Computer Science} 
\\ \itab{\textbf{Efficient Algorithms and Intractable Problems}} \tab{} \tab{Introduction to Artificial Intelligence}
\end{rSection}
\end{document}
