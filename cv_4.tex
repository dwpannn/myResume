
\documentclass{resume} 

\usepackage[left=0.25in,top=0.2in,right=0.25in,bottom=0.2in]{geometry} 
\newcommand{\tab}[1]{\hspace{.2667\textwidth}\rlap{#1}}
\newcommand{\itab}[1]{\hspace{0em}\rlap{#1}}
\name{Mingyang Pan} 
\address{909-859-5850 \\ \textbf{ympan99@gmail.com} \\ \textbf{github.com/dwpannn }\\ linkedin.com/in/wwdqd} 



\begin{document}

%----------------------------------------------------------------------------------------
%	EDUCATION SECTION
%----------------------------------------------------------------------------------------

\begin{rSection}{Education}

{\bf University of California, Berkeley} \hfill {\em May 2018 - \textbf{May 2021}} 
\\ Computer Science, B.A.


\end{rSection}
%----------------------------------------------------------------------------------------
%	TECHNICAL SKILLS SECTION
%----------------------------------------------------------------------------------------

\begin{rSection}{Technical Skills}

\begin{tabular}{ @{} >{\bfseries}l @{\hspace{6ex}} l }
Programming Languages &  \textbf{Java}, \textbf{Go}, J2EE, Ruby(on Rails), \textbf{Python}, C \\
Amazon Web Services(AWS) & AWS Certified Developer-Associate \emph{(Verification\#: ZD27HPCKEFQ11R33)} \\
Back-end Development & Spring, \textbf{Postgre DB}, \textbf{gRPC}, \textbf{REST API}, \textbf{Kafka microservices}
\end{tabular}

\end{rSection}

%----------------------------------------------------------------------------------------
%	WORK EXPERIENCE SECTION
%----------------------------------------------------------------------------------------

\begin{rSection}{Work Experience}

\begin{rSubsection}{Hewlett Packard Enterprise(HPE)}{\em June 2021 - Present}{\textbf{Software Engineer - dHCI/HCI Storage Manager, Greenlake Cloud Services}}{San Jose, California}{}
\item Worked in two scrum teams to handle urgent escalation tickets and QA-reported release tickets in agile cycles. Addressed the underlying issues by making critical fixes in the Java codebase on release branches.
\item Added multi-cluster support for dHCI storage solutions by designing and making data model changes and adding internal APIs. Enabled more scalable deployment options with 200+ nodes with a single dHCI instance. 
\item Picked up Golang on the fly. Designed and built new features from scratch for 2 cloud native Go based Kafka microservices. Implemented gRPC clients and APIs for setup orchestration.


\end{rSubsection}
\begin{rSubsection}{Hewlett Packard Enterprise(HPE)}{\em January 2021 - May 2021}{\textbf{Software Engineering Co-op - Disaggregated Hyperconverged Infrastructure(dHCI)}}{Remote}{}
\item Worked with respective code owners to refactor code in 6 repositories to bring up the language level to Java 11. Resolved Gradle dependencies in all the modules. Changed the implementation of the keystore to use accept the new PKCS12 type.
\item Implemented package level custom stress tests and persistence tests and performed heap memory analysis to verify performance. In heavy workflows like VM migration, performance was improved up to 10\%.




\end{rSubsection}

\begin{rSubsection}{Hewlett Packard Enterprise(HPE)}{\em May 2020 - August 2020}{\textbf{Software Engineering Intern - Disaggregated Hyperconverged Infrastructure(dHCI)}}{Remote}{}
\item Leveraged VMware Automation API to achieve vCenter minor update automation for HPE Nimble dHCI Intelligent 1-Click Updates feature in on-prem Java codebase. 
\item Deployed Swagger Codegen to generate client SDKs at runtime in order to further automate update-related VMware API consumption. Eliminated the need to manually model JSON response body from REST API calls in the backend.
\item Reduced update workflow time by 16.7\% for each server within the dHCI environment. Project branch was merged with the release branch at the end of internship and shipped in the next release vehicle. 


\end{rSubsection}





%------------------------------------------------



\end{rSection}

\begin{rSection}{Selected Projects}

%------------------------------------------------

\begin{rSubsection}{Web application demonstrating one to two bit oblivious transfer theorem.}{\em February 2019 - May 2019}{\textbf{GitHub: github.com/dwpannn/oblivious-theorem}}{}
\item A web application based on Ruby on Rails to resolve the issue of people feeling awkward when confessing or talking to others' about personal feelings. Namely, this prototype solves the problem of not knowing whether your crush likes you or not. The logic behind the operation is one-to-two bit oblivious transfer theorem. 
\end{rSubsection}

%------------------------------------------------

\begin{rSubsection}{Java application based on refined Huffman coding algorithm}{\em March 2019}{\textbf{GitHub: github.com/dwpannn/huffmanCoding}}{}
\item An application uses truncated tree structure to shorten the length of the binary string returned. Increased encoding efficiency over traditional tree structure by over 25\%.
\end{rSubsection}
\end{rSection}
\end{document}
